\documentclass{article}
\usepackage{graphicx} % Required for inserting images
\usepackage[LGR,T1]{fontenc}
\usepackage[utf8]{inputenc} 
\usepackage{blindtext}
\usepackage{soul}
\usepackage{xcolor}
\usepackage{hyperref}
\hypersetup{
    colorlinks=true,
    linkcolor=black,
    filecolor=black,      
    urlcolor=black,
    pdftitle={Domain Model v0.2 alt},
    pdfpagemode=FullScreen,
    }

\urlstyle{same}

\begin{document}


\graphicspath{{C:/Users/ΘΩΜΑΣ/Downloads/recoverylogo.png}}



\begin{figure}
    \centering
    \includegraphics[width=0.5\linewidth]{recoverylogo.png}

\end{figure}


\title{\Huge{\textbf{Domain Model v0.2}}}
\date{}
\maketitle

\newcommand{\textgreek}[1]{\begingroup\fontencoding{LGR}\selectfont#1\endgroup}

\center{\large{\textgreek{Τμήμα Μηχανικών Ηλεκτρονικών Υπολογιστών και Πληροφορικής Πανεπιστήμιο Πατρών}}}

\newpage
\flushleft
\section*{\textgreek{Μέλη ομάδας:}}
\author{\textgreek{Ελένη Θανοπούλου ΑΜ:1072602 5ο έτος}} \newline
\author{\textgreek{Αργυρώ Καμαρινού ΑΜ:1072511 5ο έτος}} \newline
\author{\textgreek{Νικολάου Θωμάς ΑΜ:1072590 5ο έτος}} \newline
\author{\textgreek{Θεόδωρος Παπαπαναγιώτου ΑΜ:1059644 7ο έτος}} \newline
\author{\textgreek{Ορφέας Πουργουρίδης ΑΜ:1069664 5ο έτος}} \newline
\newline
\newline
\section*{\textgreek{Συμμετείχαν:}}
\author{\textgreek{Ελένη Θανοπούλου} Reviewer} \newline
\author{\textgreek{Αργυρώ Καμαρινού} Reviewer} \newline
\author{\textgreek{Νικολάου Θωμάς} Contributor} \newline
\author{\textgreek{Θεόδωρος Παπαπαναγιώτου} Contributor} \newline
\author{\textgreek{Ορφέας Πουργουρίδης} Reviewer} \newline
\newline
\newline
\newline
\textgreek{Για την συγγραφή του τεχνικού κειμένου έχουν χρησιμοιποιηθεί τα εργαλεία:}
\begin{itemize}
    \item Google Docs\textgreek{ για την συγγραφή και την επίβλεψη από τα υπόλοιπα μέλη της ομάδας.}
    \item Overleaf\textgreek{ για την τελειοποίηση του αρχείου σε }\LaTeX.
    \item draw.io \textgreek{για τον σχεδιασμό του} domain model.
\end{itemize}

\newpage
\section*{\textgreek{Σύντομη περιγραφή των κλάσεων}}
\textbf{Actor:} \textgreek{Η κλάση} Actor \textgreek{είναι μια αφηρημένη} (abstract) \textgreek{κλάση που ορίζει τις διάφορες κοινές ιδιότητες και μεθόδους που έχουν μεταξύ τους οι χειριστές της εφαρμογής. Ουσιαστικά παρέχει την βάση για τη δημιουργία υποκλάσεων ανάλογα με τον τύπο του χειριστή.}
\newline
\newline

\textbf{\textgreek{Ιδιότητες:}}
\begin{itemize}
    \item username: String
    \item role: String
    \item urrentChatRoom: ChatRoom | null
\end{itemize}
\textbf{\textgreek{Μέθοδοι:}}
\begin{itemize}
    \item sendMessage(message: String, chatRoom: ChatRoom)
    \item joinChatRoom(chatRoom: ChatRoom)
    \item leaveChatRoom(chatRoom: ChatRoom)
\end{itemize}

\textbf{User:} \textgreek{Η κλάση} User \textgreek{είναι υποκλάση της} Actor \textgreek{και αντιπροσωπεύει τα άτομα που επιθυμούν απεξάρτηση και χρησιμοποιούν την εφαρμογή. Οι ιδιότητες και οι μέθοδοι της κλάσης αυτής επιτρέπουν στον κάθε χρήστη να μπορεί να αλληλεπιδράσει με τα φόρουμ, τις ομαδικές αίθουσες συνομιλίας, το} “streak counter”, \textgreek{το προσωπικό ημερολόγιο και άλλες λειτουργικότητες της εφαρμογής.}
\newline
\newline

\textbf{\textgreek{Ιδιότητες}}
\begin{itemize}
    \item anonymous: Boolean
\end{itemize}
\textbf{\textgreek{Μέθοδοι:}}
\begin{itemize}
    \item toggleAnonymous()
    \item sendPrivateMessage(receiver: User | Therapist, message: String)
    \item reportMessage(message: Message, chatRoom: ChatRoom)
    \item reportUser(user: User, chatRoom: ChatRoom)
\end{itemize}

\textbf{Therapist:} \textgreek{Η κλάση} Therapist \textgreek{είναι υποκλάση της} Actor \textgreek{και αντιπροσωπεύει τους συμβούλους ψυχικής υγείας της εφαρμογής. Κάποιες πρόσθετες ιδιότητες της κλάσης αυτής είναι τα προσόντα και οι τομείς εξειδίκευσης των συμβούλων.}
\newline
\newline

\textbf{\textgreek{Ιδιότητες}}
\begin{itemize}
    \item qualifications: Array<String>
    \item specialties: Array<String>
\end{itemize}
\textbf{\textgreek{Μέθοδοι:}}
\begin{itemize}
    \item guideUser(receiver: User, message: String)
    \item sendPrivateMessage(receiver: User | Therapist, message: String)
    \item reportMessage(message: Message, chatRoom: ChatRoom)
    \item reportUser(user: User, chatRoom: ChatRoom)
\end{itemize}

\textbf{Volunteer:} \textgreek{Η κλάση} Volunteer \textgreek{είναι υποκλάση της} Actor \textgreek{και αντιπροσωπεύει τους εθελοντές που προσφέρουν υποστήριξη και συνεισφέρουν περιεχόμενο στην εφαρμογή. Περιέχει κάποιες πρόσθετες ιδιότητες όπως τους τύπους της προσφερόμενης υποστήριξης καθώς και μεθόδους σχετικές με τη συνεισφορά πόρων.}
\newline
\newline

\textbf{Admin:} \textgreek{Η κλάση} Admin \textgreek{είναι υποκλάση της} Actor \textgreek{και αντιπροσωπεύει τον διαχειριστή της εφαρμογής. Οι ιδιότητες αυτής της κλάσης έχουν να κάνουν με τις ρυθμίσεις του συστήματος ενώ οι μέθοδοι της με την διαχείριση χρηστών και τη διαμόρφωση του συστήματος.}
\newline
\newline

\textbf{\textgreek{Ιδιότητες}}
\begin{itemize}
    \item
\end{itemize}
\textbf{\textgreek{Μέθοδοι:}}
\begin{itemize}
    \item createChatRoom(roomDetails: Object): boolean
    \item editChatRoom(roomId: String, updatedDetails: Object): boolean
    \item deleteChatRoom(roomId: String): boolean
\end{itemize}

\textbf{Moderator:} \textgreek{Η κλάση} Moderator \textgreek{είναι υποκλάση της} Actor \textgreek{και αντιπροσωπεύει τους επόπτες περιεχομένου της εφαρμογής. Περιέχει ιδιότητες και μεθόδους που υλοποιούν λειτουργικότητες σχετικές με εργαλεία παρακολούθησης χρηστών, έλεγχο περιεχομένου και επιβολή των κανόνων της κοινότητας.}
\newline
\newline

\textbf{\textgreek{Ιδιότητες}}
\begin{itemize}
    \item
\end{itemize}
\textbf{\textgreek{Μέθοδοι:}}
\begin{itemize}
    \item monitorChatRoom()
    \item interveneInappropriateBehavior()
    \item deleteMessage(message: Message, chatRoom: ChatRoom)
    \item muteUser(user: User)
    \item banUser(user: User)
\end{itemize}

\textcolor{red}{\textbf{Streak:} \textgreek{Η κλάση} Streak \textgreek{αναφέρεται στον συνεχόμενο χρόνο για τον οποίο ο χρήστης έχει παραμείνει νηφάλιος.Περιέχει ιδιότητες όπως την αρχική ημερομηνία που ο άρχισε την απεξάρτηση καθώς και την ουσία/συμπεριφορά από την οποία θέλει να απεξαρτηθεί.}}
\newline
\newline

\textcolor{red}{\textbf{\textgreek{Ιδιότητες}}
\begin{itemize}
    \item StreakRecord: datetime
    \item StreakSubstance: String
    \item StreakId : int
\end{itemize}}

\textbf{StreakCounter:} \textgreek{Η κλάση} StreakCounter \textgreek{αποτελεί το σύστημα καταγραφής προόδου του χρήστη όσον αφορά την αποχή του από μια εθιστική ουσία ή συμπεριφορά. Περιλαμβάνει ιδιότητες και μεθόδους που υλοποιούν μηχανισμούς σχετικά με την παρακολούθηση της διάρκειας νηφαλιότητας του χρήστη, την απονομή βραβείων} (chip) \textgreek{ανάλογα με το ορόσημο που έχει επιτύχει \st{καθώς και την καταγραφή τυχόν υποτροπών.}}
\newline
\newline
\textbf{\textgreek{Ιδιότητες}}
\begin{itemize}
    \item CurrentStreaks :Array<Streak>
    \item StreakMngmntChoices : boolean
    \item Calendar : (StartingDate)
    \item Substances : Array <String>
    \item StartingDate : Array <date>
    \item CurrentRecord : Array <date>
    \item CurrentLevel : String
    \item Levels : Array <String>
    \item Motives : Array <String>
\end{itemize}
\textbf{\textgreek{Μέθοδοι:}}
\begin{itemize}
    \item getCurrentStreaks(): ArrayList<CurrentStreaks>
    \item manageStreaks(): StreakMngmntChoices
    \item addStreak(Streak : Streak)
    \item delStreak(Streak : Streak)
    \item delCurrentRecord(CurrentRecord : CurrentRecord)
    \item delCurrentLevel(CurrentLevel : CurrentLevel)
    \item getSubstances() : Substances
    \item setSubstance() : Substances
    \item getCalendar() : Calendar
    \item setStartingDate(StartingDate : Startingdate)
    \item setCurrentRecord(CurrentRecord : CurrentRecord)
    \item setCurrentLevel(CurrentLevel : CurrentLevel)
    \item getMotives(): text
    
\end{itemize}

\textcolor{red}{\textbf{\textgreek{Υποτροπές:}} \textgreek{Η κλάση υποτροπες αναφέρεται στις υποτροπές που ο χρήστης έχει ήδη καταγράψει. Περιλαμβάνει ιδιότητες όπως την ημερομηνία της υποτροπής και το} StreakId \textgreek{για το} Streak \textgreek{το οποίο ο χρήστης έσπασε.}}
\newline
\newline

\textcolor{red}{\textbf{\textgreek{Ιδιότητες}}
\begin{itemize}
    \item RelapseDate : datetime
    \item StreakId : int
\end{itemize}}

\textbf{\textgreek{Προηγούμενες Υποτροπές:}} \textgreek{Η κλάση προηγούμενες υποτροπές αποτελεί το σύστημα καταγραφής υποτροπών από το χρήστη. Περιλαμβάνει ιδιότητες όπως προηγούμενες υποτροπές του χρήστη και το Record to Beat και μεθόδους που υλοποιούν λειτουργίες σχετικά με την καταγραφή υποτροπών από το χρήστη.}
\newline
\newline

\textcolor{red}{\textbf{\textgreek{Ιδιότητες}}
\begin{itemize}
    \item PreviousRelapses : (Relapses)
    \item Calendar : RelapseDate
    \item CurrentRecord : datetime
    \item CurrentLevel : String
    \item RecordToBeat : datetime
\end{itemize}
\textbf{\textgreek{Μέθοδοι:}}
\begin{itemize}
    \item getPreviousRelapses():ArrayList<Relapses>
    \item getCalendar() : Calendar
    \item setRelapseDate(RelapseDate : RelapseDate)
    \item getCurrentRecord() : CurrentRecord
    \item cmprCurrentRecord() : CurrentRecord
    \item setRecordToBeat(RecordToBeat : RecordToBeat)
    \item zeroCurrentLevel(CurrentRecord : CurrentRecord)
    \item zeroCurrentLevel(CurrentLevel : CurrentLevel)
\end{itemize}}

\textbf{Feed:} \textgreek{Η κλάση} Feed \textgreek{αποτελεί την ροή περιεχομένου του φόρουμ. Μέσω των ιδιοτήτων και των μεθόδων της γίνεται διαχείριση του περιεχομένου του φόρουμ δηλαδή των αναρτήσεων των χρηστών και των άρθρων από ιστοσελίδες}
\newline
\newline

\textbf{\textgreek{Ιδιότητες}}
\begin{itemize}
    \item feedType: String
    \item posts: Array<Post>
    \item articles: Array<Article>
\end{itemize}
\textbf{\textgreek{Μέθοδοι:}}
\begin{itemize}
    \item displayFeedPosts()
    \item displayFeedArticles()
    \item getAllPosts(): Array<Post> 
    \item getAllArticles(): Array <Article>
    \item addPost(postDetails: Object)
    \item addArticle(articleDetails: Object)
    \item removePost(post: Post)
    \item removeArticle(article: Article)
\end{itemize}

\textbf{Post:} \textgreek{Η κλάση} Post \textgreek{αντιπροσωπεύει τις αναρτήσεις που δημιουργούνται από τους χρήστες στο φόρουμ. Περιλαμβάνει ιδιότητες σχετικά με τον δημιουργό της ανάρτησης, το περιεχόμενο, τις αντιδράσεις και τα σχόλια της ανάρτησης και μεθόδους σύμφωνα με τις οποίες μπορεί να γίνει αλληλεπίδραση με μια ανάρτηση.}
\newline
\newline

\textbf{\textgreek{Ιδιότητες}}
\begin{itemize}
    \item postId: String
    \item author: String
    \item content:
    \item reactions: Array<>
    \item comments: Array<>
\end{itemize}
\textbf{\textgreek{Μέθοδοι:}}
\begin{itemize}
    \item addPost()
    \item addReaction()
    \item addComment()
    \item sendFriendRequest()
    \item reportPost()
    \item deletePost()
    \item banUser()
\end{itemize}

\textbf{Website:} \textgreek{Αντιπροσωπεύει την ιστοσελίδα που μπορούν να ακολουθούν οι χρήστες για να λαμβάνουν άρθρα στο} feed \textgreek{του φόρουμ. Περιλαμβάνει ιδιότητες όπως το είναι το όνομα και η περιγραφή της ιστοσελίδας και μεθόδους σχετικά με την ανάκτηση άρθρων απ΄την ιστοσελίδα μέσω} API.
\newline
\newline


\textbf{Article:} \textgreek{Αντιπροσωπεύει ένα άρθρο που έχει ληφθεί από εξωτερικές ιστοσελίδες και περιέχει ιδιότητες όπως είναι η διεύθυνση} URL \textgreek{του άρθρου, ο τίτλος του, η περιγραφή του, η πηγή του κλπ.}
\newline
\newline

\textbf{\textgreek{Ιδιότητες}}
\begin{itemize}
    \item title: string
    \item description: string
    \item carticleLink: string
    \item authorName: string
\end{itemize}
\textbf{\textgreek{Μέθοδοι:}}
\begin{itemize}
    \item 
\end{itemize}

\textbf{ExternalLink;} \textgreek{Η κλάση} ExternalLink \textgreek{αντιπροσωπεύει έναν εξωτερικό σύνδεσμο για την πλοήγηση των χρηστών σε περιεχόμενο εκτός της πλατφόρμας.}
\newline
\newline

\textbf{\textgreek{Ιδιότητες}}
\begin{itemize}
    \item externalLink: String
\end{itemize}

\textbf{Personal\textcolor{red}{Diary:}}\textgreek{Η κλάση} Personal\textcolor{red}{Diary} \textgreek{αποτελεί το προσωπικό ημερολόγιο του χρήστη. Περιλαμβάνει \st{ιδιότητες όπως την ημερομηνία συγγραφής και το περιεχόμενο ενός εγγράφου και} μεθόδους σχετικές με την δημιουργία, την ανάγνωση, την επεξεργασία και τη διαγραφή εγγράφων ημερολογίου.}
\newline
\newline

\textbf{\textgreek{Ιδιότητες}}
\begin{itemize}
    \item entries: Array<Entry>
\end{itemize}
\textbf{\textgreek{Μέθοδοι:}}
\begin{itemize}
    \item createEntry(entryDetails: Object): Entry
    \item editEntry(entry: Entry, updatedDetails: Object)
    \item deleteEntry(entry: Entry)
    \item getAllEntries(): Array<Entry>
\end{itemize}

\textcolor{red}{\textbf{Entry:} \textgreek{Η κλάση} Entry \textgreek{αντιπροσωπεύει μια μεμονωμένη καταχώρηση σε ένα προσωπικό ημερολόγιο. Χρησιμεύει ως ένας δομημένος τρόπος αποθήκευσης και διαχείρισης καταχωρήσεων ημερολογίου.}}
\newline
\newline

\textcolor{red}{\textbf{\textgreek{Ιδιότητες}}
\begin{itemize}
    \item entryId: String
\end{itemize}
\textbf{\textgreek{Μέθοδοι:}}
\begin{itemize}
    \item updateDetails(updatedDetails: Object)
\end{itemize}}


\textbf{ChatRoom:} \textgreek{Η κλάση} ChatRoom \textgreek{αντιπροσωπεύει μια αίθουσα συνομιλίας εντός της πλατφόρμας όπου χρήστες μπορούν να συνδέονται για να αλληλεπιδρούν και να υποστηρίζουν ο ένας τον άλλον. Περιλαμβάνει ιδιότητες όπως το όνομα της αίθουσας συνομιλίας, την περιγραφή της, τα μέλη της, μια λίστα με τις προσωπικές συζητήσεις μεταξύ των μελών της κλπ.}
\newline
\newline

\textbf{\textgreek{Ιδιότητες}}
\begin{itemize}
    \item roomId: String
    \item name: String
    \item capacity: Number
    \item currentParticipants: Array<User>
    \item messages: Array<Message>
    \item isFull: Boolean
    \item isRestricted: Boolean
    \item isMonitored: Boolean
\end{itemize}
\textbf{\textgreek{Μέθοδοι:}}
\begin{itemize}
    \item addParticipant(user: User): boolean
    \item removeParticipant(user: User): boolean
    \item postMessage(message: Message)
    \item listMessages(): Array<Messages>
    \item checkAvailability(): boolean
    \item setMonitored(isMonitored: Boolean)
\end{itemize}

\textbf{PrivateDiscussion:} \textgreek{Η κλάση} PrivateDiscussion \textgreek{ανήκει στο} ChatRoom \textgreek{και αντιπροσωπεύει την προσωπική/ιδιωτική συζήτηση μεταξύ των μελών της αίθουσας συνομιλίας.}
\newline
\newline

\textbf{\textgreek{Ιδιότητες}}
\begin{itemize}
    \item discussionId: String
    \item participants: Array<User>
    \item messages: Array<Message>
\end{itemize}
\textbf{\textgreek{Μέθοδοι:}}
\begin{itemize}
    \item addMessage(message: Message)
    \item removeParticipant(user: User)
    \item getMessages(): Array<Message>
\end{itemize}

\textbf{Message:} \textgreek{Η κλάση} Message \textgreek{αντιπροσωπεύει τα μηνύματα μεταξύ των χρηστών μέσα στις λειτουργίες του φόρουμ και των αιθουσών συνομιλιών} (chat rooms) \textgreek{της εφαρμογής. Οι ιδιότητες περιλαμβάνουν τον αποστολέα, τον παραλήπτη, το περιεχόμενο του μηνύματος και την χρονική στιγμή που δημιουργήθηκε. Οι μέθοδοι έχουν να κάνουν με την αποστολή, την λήψη και την διαγραφή μηνυμάτων.}
\newline
\newline

\textbf{\textgreek{Ιδιότητες}}
\begin{itemize}
    \item messageId: String
    \item content: String
    \item sender: String
    \item timestamp: Date
\end{itemize}
\textbf{\textgreek{Μέθοδοι:}}
\begin{itemize}
    \item edit(newContent: String)
    \item delete()
\end{itemize}

\textcolor{red}{\textbf{Notification:} \textgreek{Η κλάση} Notification \textgreek{αντιπροσωπεύει την αποστολή ειδοποιήσεων προς τον χρήστη.}}
\newline
\newline

\textcolor{red}{
\textbf{\textgreek{Ιδιότητες}}
\begin{itemize}
    \item notificationTitle(): string
    \item notificationDescription(): string
\end{itemize}
\textbf{\textgreek{Μέθοδοι:}}
\begin{itemize}
    \item sendNotification()
    \item cancelNotification()
\end{itemize}
}
\newpage

\begin{figure}
    \centering
    \includegraphics[width=1\linewidth]{domainmodel.png}
    \caption{To domain model}
    \label{fig:enter-label}
\end{figure}





\end{document}
\begin{figure}
    \centering
    \includegraphics[width=0.5\linewidth]{recoverylogo.png}
    \caption{Enter Caption}
    \label{fig:enter-label}
\end{figure}